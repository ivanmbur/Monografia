\documentclass[12pt]{report}

\usepackage[utf8]{inputenc}
\usepackage{graphicx}
\usepackage{amssymb}
\usepackage{amsmath}
\usepackage{amsthm}
\usepackage{hyperref}

\graphicspath{ {images/} }

\theoremstyle{definition}
\newtheorem{example}{Example}[section]
\newtheorem{definition}[example]{Definition}
\newtheorem{theorem}[example]{Theorem}
\newtheorem{corollary}[example]{Corollary}
\newtheorem{notation}[example]{Notation}

\DeclareMathOperator{\tr}{tr}
\DeclareMathOperator{\Aut}{Aut}
\DeclareMathOperator{\ev}{ev}
\DeclareMathOperator{\im}{Im}
\DeclareMathOperator{\re}{Re}
\DeclareMathOperator{\Span}{span}
\DeclareMathOperator{\id}{id}

\allowdisplaybreaks

\title 
{
	{KMS states and Tomita-Takesaki Theory}\\
	{\large Universidad de los Andes}\\
	{\includegraphics{logo.png}}	
}
\author{Iván Mauricio Burbano Aldana\\[1cm]{\small Advised by: Prof. Andrés Fernando Reyes Lega}}

\begin{document}

\pagenumbering{Roman}

\maketitle

\tableofcontents

\chapter{Introduction}
\pagenumbering{arabic}
One of the most important problems in modern physics is that of the mathematical formulation of quantum field theory. Although used successfully throughout the community to study the most relevant problems of particle physics, solid state physics, cosmology, and the dark sector, among others, it still lacks a complete rigorous mathematical formulation. Indeed, divergences and ill-defined symbols rid the whole theory. As physicists, it is not only our duty to give predictions of the physical world but to understand its working principles. In particular, a complete understanding of these must be of a logical nature. It is our believe that mathematics serves this purpose, especially in the cases where those who claim to have an intuitive understanding are often wrong. The purpose of this monograph is to do a bibliographical revision of a result in the realm of mathematical physics and in the search for the correct mathematical framework of quantum theories with infinite degrees of freedom: to every thermodynamical equilibrium quantum state there is a canonical dynamical law governing the time evolution of the system.

In chapter \ref{chp:axiom} we do a quick revision of general frameworks in classical and quantum theories. This serves mainly to establish notation and point towards some common elements these theories have which will inspire the algebraic approach present throughout this monograph. Before we delve into this unifying scheme, in chapter \ref{chp:logic} we attempt to understand the differences between these theories. We arrive at the conclusion that it has is roots in the mathematical structure of the propositions associated to a quantum theory. In particular, the structure will not be that of a boolean algebra\footnote{This departure from the mathematical framework of classical propositions is the root towards my skepticism towards those who claim to have an intuitive understanding of quantum theory.}. In chapter \ref{chp:algebra} we present the general framework of algebraic quantum theory. This is inspired on the features noted in \ref{chp:axiom}. In particular, we develop the theory of the GNS construction (which we exemplify by showing how it can aid in the calculation of entropies) and of dynamical systems, the two stepping stones in the path to the main result. In chapter \ref{chp:KMS} we study KMS states. These will be states characterized by certain analytic properties (the KMS condition) which we will interpret as those of quantum states in thermodynamic equilibrium. This will be inspired by studying the relationship between KMS states and Gibbs states (the canonical ensemble) in finite dimensional quantum systems. Chapter \ref{chp:tomita} develops Tomita-Takesaki theory. This theory will yield the mathematical objects that appear in the main result of this monograph. We follow the approach of \cite{Duvenhage1999} and \cite{Rieffel1977} to avoid encountering unbounded operators and domain issues. Finally, in chapter \ref{chp:final} we gather the partial result obtained in \ref{chp:algebra}, \ref{chp:KMS}, and \ref{chp:tomita} and develop the final result which amounts to the connection between KMS states and Tomita-Takesaki theory.  

   

\chapter{Classical and Quantum Mechanics as Probability Theories}\label{chp:axiom}
This chapter shows how both classical and quantum mechanics are probability theories. This is not intended as an aximatization of these theories. Indeed the reader is assumed to be comfortable with these physical theories as well as the basic mathematical concepts of measure theory and functional analysis.

\section{Classical Mechanics}

The setting of classical mechanics is usually a locally compact Hausdorff space $X$. We consider the states of maximal knowledge (or pure states) to be the elements of $X$. Likewise, observables take the form of functions on $X$. We call the points of $X$ states of maximal knowledge because we interpret $f(p)$ as the value of the obsevable $f$ in the state $p\in X$. Moreover, given that in principle we could make the value of an observable as precise as we want by improving our knowledge of the state, observables are elements of the set of continuous functions $C(X)$\footnote{It doesn't matter whether we consider them real or complex at this stage.}. Nonetheless the purpose of statistical mechanics is to treat systems in which total knowledge of a state is not practically posible. Instead we consider a probability measure\footnote{Along with a $\sigma$-algebra which we won't mention explicitely to keep the notation simple but should always be kept in mind.} which assigns to every measurable subset of $X$ a probability of the system's state being in it. We may define the expected value of an observable $f\in C(X)$ through a probability measure $P$ by 
\begin{equation}
\langle f \rangle_P = \int fdP.
\end{equation}  
Notice that an element $p\in X$ can also be though as a probability measure by using the Dirac measure $\delta_p$ which assigns $1$ to a set if it contains $p$ and $0$ otherwise. Indeed for every element $p \in X$ and observable $f \in C(X)$ we have $\langle f \rangle_{\delta_p} = f(p)$. This motivates us to broaden the definition of states to the probability measures on $X$. We will call Dirac measures (or equivalently the points in $X$) pure states.

This definition of state proves to be very helpful for the discussion of ensembles. Whenever the description of the state of a system as a pure state is not feasible, we may consider the set of outcomes $Y$ of measurements we may perform on the system. Every element of $Y$ gives us information of the system in the form of a finite measure. We may define an ensemble as the mapping from $Y$ into the set of finite measures on $X$. Through normalization of finite measures every ensemble yields a mapping from $Y$ into the set of states and we define the accesible (pure) states of an element $y\in Y$ to be the support of the corresponding state. Although the construction of an ensemble is in general a difficult task, for systems in statistical equilibrium\footnote{These are systems whose state does not change in time. We refer to the equilibrium as statistical because it may be that the pure state of the system is changing in time but noticing these changes is not feasible for us.} there are many standard procedures. In the case of these type of systems we define the partition function $Z:Y\to \mathbb{R}^+_0$ by assigning to every element $y$ the measure of $X$ given by the ensemble evaluated at $y$.

\begin{example}
In many physical systems the space of pure states has a natural notion of size which we may represent by giving it the structure of a measure space $(X,\mathcal{A},\mu)$ where $\mathcal{A}$ contains the Borel $\sigma$-algebra\footnote{Usually we take a countable set with the counting measure but another example would be a phase space with the Liouville measure. In the latter, it is common that $H^{-1}(y)$ is a set of measure zero so we actually have to take $X=H^{-1}(y)$ with the appropiate induced measure.} . We may consider $Y = \mathbb{R}$ to be the set of energy outcomes. If $H:X\rightarrow \mathbb{R}$ is a measurable function taking the interpretation of energy we define the microcanonical ensemble to be the mapping $y\mapsto \mu_y$ where $\mu_y(\Sigma) = \mu(\Sigma\cap H^{-1}(y))$ for all measurable $\Sigma$. The set $H^{-1}(y)$ is the set of accesible states and $\mu_y(\Sigma)$ measures the amount of pure states in $\Sigma$ which are accesible. Notice that the normalization of $\mu_y$ yields a state $P_y$ which assigns a uniform probability measure to $X$. This is called the equal a priori probabilities postulate. In this ensemble the partition function $Z(y) = \mu_y(X) = \mu(H^{-1}(y))$ is just the amount of accesible states. This ensemble is usually used to describe systems with constant energy and a fixed number of particles.
\end{example}

\begin{example}
Consider again a measure space $(X,\mathcal{A},\mu)$ but let $Y=\mathbb{R}^+$ be the set of inverse temperatures of the system. If we have an energy function $H:X\rightarrow \mathbb{R}$ such that $x\rightarrow \exp(-y H(x))$ is integrable for all $y \in Y$ the canonical ensemble assigns to every inverse temperature $y$ a finite measure $\mu_y$ by
\begin{equation}
\mu_y(\Sigma)=\int_\Sigma e^{-y H(x)}d\mu(x)
\end{equation}
for all measurable sets $\Sigma$. This ensemble is usually used to describe systems with a fixed number of particles in thermal equilibrium with a heat bath. Note that we could add to the description of the system the heat bath and we would be able to in principle use the microcanonical ensemble. The difficulty lies in that generally the counting of accesible states is more difficult than the application of the canonical ensemble.
\end{example}

Note that both of the ensembles discussed have images consisting of absolutely continuous measures $\mu_y$ with respect to the notion of size $\mu$. The same is true for the induced states $P_y$. Moreover the Lebesgue-Radon-Nikodým derivative exits and we define the entropy of the ensemble in the state $P_y$ by\footnote{In general if we start from a decomposible $(X,\mathcal{A},\mu)$ and have an ensemble which yields absolutely continuous measures with respect to $\mu$ we can define entropy in this fashion. In particular, if $\mu$ comes from the Daniel extension of a positive linear functional the space is decomposible.}
\begin{equation}
S(P_y)=-\int_{supp(P_y)} \log\left(\frac{dP_y}{d\mu}\right)\frac{dP_y}{d\mu}d\mu.
\end{equation}
One can check that in the microcanonical ensemble 
\begin{equation}
\frac{dP_y}{d\mu}(x)=\frac{\chi_{H^{-1}(y)}(x)}{Z(y)}
\end{equation}
and in the canonical ensemble 
\begin{equation}
\frac{dP_y}{d\mu}(x)=\frac{\exp(-yH(x))}{Z(y)}. 
\end{equation}
In the case we have a state of maximal knowledge $\delta_p$ we may collapse $X$ to $(\{p\},\{\{p\},\emptyset\},\delta_p)$ and define an ensemble $y\mapsto \delta_p$. Such an ensemble has zero entropy.

\section{Quantum Mechanics}\label{sec:QM}

The setting of quantum mechanics is a separable Hilbert space $\mathcal{H}$. In this case the states are the non-negative self-adjoint operators of unit trace on $\mathcal{H}$ (called density operators) and observables take the form of self-adjoint operators on $\mathcal{H}$. The possible outcomes of an observable $A$ are the elements of its spectrum and, if $P_A$ is the unique projection-valued measure such that $A=\int id_\mathcal{H}dP_A$ given by the spectral theorem, we have that the probability of the measurement of the observable $A$ yielding a value in the measurable subset $E\subseteq\mathbb{R}$ in the state $\rho$ is $tr(P_A(E)\rho)$. One can check that given this way of measuring probabilities we have that the expected value of an observable $A$ in the state $\rho$ is
\begin{equation}
\langle A\rangle_\rho = tr(A\rho).
\end{equation}
Moreover we define the entropy of a state $\rho$ to be
\begin{equation}
S(\rho)=-tr(\log(\rho)\rho).
\end{equation}
Inspired by the classical case we define a pure state $\rho_\psi$ to be orthogonal projection on the span of $\psi$ for $\psi\in\mathcal{H}$ of unit norm. Although such a state has null entropy as in the classical case
\begin{equation}
S(\rho_\psi)=-tr(\log(\rho_\psi)\rho_\psi)=-\langle\psi,\log(1)\psi\rangle = 0,
\end{equation} 
we can't in general associate to an observable $A$ a definite outcome unless $\psi$ is an eigenvector of $A$ corresponding to an eigenvalue $\lambda$ where we have 
\begin{equation}
tr(P_A(\{\lambda\})\rho_\psi)=\langle\psi,P_A(\{\lambda\})\psi\rangle=\langle\psi,\psi\rangle=\|\psi\|^2=1. 
\end{equation}
Notice that we haven't inspired the definition of a state like we did in the classical case. The connection between states and probability is given through the study of quantum logic in the next chapter.

\chapter{Quantum Probability}\label{chp:logic}
The previous chapter showed that there is a dictionary to understand in a very similar language quantum and classical theories. Before we develop the language of operator algebras to make this similarity more concrete we will first show how this two theories are different. To do this we will study the logical structure of quantum mechanics and show that it isn't boolean.

\section{EPR paradox}

Einstein, Podolsky and Rosen examined the completeness of quantum mechanics in their famous 1935 paper \cite{Einstein1935}. They considered that an element of physical reality was one whose outcome in a measurement could be predicted without actually performing the experiment. They defined that a physical theory was complete if to every element of physicial reality there corresponded an object in the theory. One can prove that in quantum mechanics two observables $A$ and $B$\footnote{We will assume them to be bounded to avoid technical difficulties since our example will be finite dimensional.} satisfy for every state $\rho$ the Heisenberg uncertainty relation
\begin{equation}
\Delta_\rho A\Delta_\rho B\geq\frac{1}{2}|\langle\left[A,B\right]\rangle_\rho|
\end{equation}
where $\left[A,B\right]=AB-BA$ is the commutator and $\Delta_\rho A=\sqrt{\langle A^2\rangle-\langle A\rangle^2}$. Therefore either quantum mechanics is incomplete or two non-commuting observables cannot have a simultaneus physical reality. Assuming that quantum mechanics is indeed complete we are forced to accept that two non-commuting observables don't have a simultaneus physical reality.

\begin{example}
To descibe the polarization of a photon we may consider the hilbert space $\mathbb{C}^2$. We assign to the observable "\textit{the photon is linarly polarized at an angle $\theta$ (1 means that this is the case and 0 that it isn't)}" the operator $P(\theta)$ the orthogonal projection to the span of $|\theta\rangle=\cos(\theta)(0,1)+\sin(\theta)(1,0)$. Therefore the vector $|\theta\rangle$ gives us the state in which the photon is certain to have linear polarization at angle $\theta$. Now consider a system with two photons in a state $\rho_{\psi}$ where $\psi=\frac{1}{\sqrt{2}}\left(|0\rangle\otimes|\pi/2\rangle-|\pi/2\rangle\otimes|0\rangle\right)$. We may prepare such a system by allowing a Calcium atom to decay into two photons and waiting till the photons are far apart. It's easy to see that $\psi=\frac{1}{\sqrt{2}}\left(|\pi/4\rangle\otimes|3\pi/4\rangle-|3\pi/4\rangle\otimes|\pi/4\rangle\right)$. Therefore if we measure that the first photon has horizontal polarization we know the second one has a vertical polarization and if we measure that the first one has a polarization at an angle $\pi/4$ we know the second one has an angle of $3\pi/4$. But since the photons are far apart, measurements on the first one cannot affect the second one. Therefore both states $|\pi/2\rangle$ and $|3\pi/4\rangle$ describe the same physical reality and we are forced to conclude that $P(\pi/2)$ and $P(3\pi/4)$ have simultaneus realities. Nonetheless since $|\pi/2\rangle$ is not orthongonal to $|3\pi/4\rangle$ the two observables don't commute arriving to a contradiction. 
\end{example}

Contradictions of the type shown above due to the use of coupled systems are often refered to nowadays to the EPR paradox. They led to the notion of entanglement. We are therefore, subject to the definitions given in the paper \cite{Einstein1935} forced to the conclusion that quantum theory must be an incomplete theory.  

\section{Logic and Bell's inequalities}

We may continue EPR's agenda and try to find a complete theory of physical reality. In such a theory (much like in every physicial theory) we must be able to ask true or false questions about a physical system. Studying these questions gives us an excuse to begin the discussion of logic theory. Although previous knowledge of logic is not essential, we will use our experience from classical logic to inspire the definitions we will use.

\begin{definition}
An order relation on a set $P$ is a relation $\leq$ on $X$ which satisfies for all $p,q,r\in P$:
\begin{itemize}
\item reflexivity: $p\leq p$;
\item antisymmetry: $p\leq q$ and $q\leq p$ implies $p=q$;
\item transitivity: $p\leq q$ and $q\leq r$ implies $p\leq r$.
\end{itemize}
The pair $(P,\leq)$ is called a partially ordered set or poset.
\end{definition}

We may recognize these laws if we exchange the symbols $\leq$ for the implication symbol $\implies$. 

\begin{definition}
Let $(P,\leq)$ be a poset and $A\subseteq P$. A lower (upper) bound of $A$ is an element $p\in P$ such that for all $a \in A$ we have $p\leq a$ ($a\leq p$). An infimum (supremum) of $A$ is a lower (upper) bound $p$ of $A$ such that if $r\in P$ is a lower (upper) bound of $A$ then $r\leq p$ ($p\leq r$).     
\end{definition}

Once again, through the symbol exchange made earlier we may note that the conjunction of two propositions $p\wedge q$ is the infimum of $\{p,q\}$ and the disjunction $p\vee q$ is the supremum of $\{p,q\}$. 

\begin{theorem}
Let $(P,\leq)$ be a poset and $A\subseteq P$ such that its infimum (supremum) exists. Then the infimum (supremum) is unique. 
\end{theorem}

\begin{proof}
Suppose $p$ and $q$ are infima (suprema) of $A$. Then since $p$ is a lower (upper) bound we have $p\leq q$ ($q\leq p$). Similarly, since $q$ is a lower (upper) bound $q\leq p$ ($p\leq q$). Therefore by antisymmetry $p=q$.  
\end{proof}

\begin{notation}
Let $(P,\leq)$ be a poset and $A\subseteq P$ such that its infimum (supremum) exists. We denote the infimum (supremum) of $A$ by $\bigwedge A$ ($\bigvee A$). If $A=\{p,q\}$ then we denote $p\wedge q :=\bigwedge A$ ($p\vee q := \bigvee A$). As is common in logic literature we will now use for infimum (supremum) the term meet (join). 
\end{notation}

Now we shall list some of the algebraic properties of posets.

\begin{theorem}\label{thm:logic_algebra}
Let $(P,\leq)$ be a poset. Then for all $p,q,r\in P$:
\begin{enumerate}
\item $p\leq q$ if and only if $p=p\wedge q$ if and only if $q =p\vee q$;
\item (idempotency) $p\wedge p = p$ and $p\vee p=p$;
\item (associativity) if the meet (join) of $\{p,q\}$, $\{q,r\}$, $\{p\wedge q, r\}$ ($\{p\vee q, r\}$), $\{p,q\wedge r\}$ ($\{p,q\vee r\}$) and $\{p,q,r\}$ exists then $(p\wedge q)\wedge r=p\wedge(q\wedge r)=\bigwedge \{p, q, r\}$ ($(p\vee q)\vee r=p\vee(q\vee r)=\bigvee \{p, q, r\}$);
\item (commutativity) if the meet (join) of $\{p,q\}$ exists then $p\wedge q = q\wedge p$ ($p\vee q = q\vee p$).
\end{enumerate}
\end{theorem}
\begin{proof}
All of the statements are clear from the definitions.
\end{proof}

All of these properties are familiar from basic logic. Nevertheless, there are some properties of basic logic which we cannot prove with the definitions above and need to be added as additional properties of posets.

\begin{definition}

\begin{itemize}
\item A poset $(P,\leq)$ is said to be a lattice if for every $p,q\in P$ there exists $p\wedge q$ and $p\vee q$.
\item A lattice $(L,\leq)$ is said to be complete if for every $A\subseteq L$ there exists $\bigwedge A$ and $\bigvee A$.
\item A lattice $(L,\leq)$ is said to be distributive if for every $p,q,r\in L$ we have $p\wedge (q\vee r)=(p\wedge q)\vee (p\wedge r)$ and $p\vee (q\wedge r)=(p\vee r)\wedge (p\vee r)$.
\item A poset $(P,\leq)$ is said to be bounded if there exists $0:=\bigwedge P$ and $1:=\bigvee P$.
\item A complement of an element $p\in P$ of a bounded poset $(P,\leq)$ is an element $q\in P$ such that $p\wedge q = 0$ and $p\vee q = 1$.
\item A Boolean algebra is a distributive bounded lattice in which every element has a complement.  
\end{itemize}

\end{definition}

Comparison with usual logic shows that we usually equip propositions with the structure of a Boolean algebra. In this case complements take the interpretation of negation and are unique due to the following theorem.

\begin{theorem}\label{thm:distributive}
In a distributive bounded lattice $(L,\leq)$ elements have at most one complement.
\end{theorem}

\begin{proof}
Suppose $q$ and $r$ are complements of $p\in L$. Then
\begin{equation}
q = q\wedge 1 = q\wedge (p\vee r) =(q\wedge p)\vee(q\wedge r) = 0\vee(q\wedge r)=q\wedge r 
\end{equation}
and therefore $q\leq r$. Exchanging the roles of $q$ and $r$ one finds that $r\leq q$ and therefore by antisymmetry $q=r$.
\end{proof}

Now, we ask that the set of propositions in a complete theory of physical reality has the structure of classical propositions, that is of a Boolean algebra. Denoting the complement of a proposition $p$ by $p'$ we may consider the following logical function
\begin{equation}
f(p,q)=(p\wedge q)\vee (p' \wedge q').
\end{equation}

With the help of the algebraic properties of this structure given in theorem \ref{thm:logic_algebra} we find that for all propositions $p_1$, $p_2$, $q_1$ and $q_2$
\begin{multline}
f(p_1,q_1)\wedge (f(p_1,q_2)\vee f(p_2,q_2)\vee f(p_2,q_1)) = \\ ((p_1\wedge q_1)\vee (p'_1 \wedge q'_1))\wedge \\ ((p_1\wedge q_2)\vee (p'_1 \wedge q'_2)\vee(p_2\wedge q_2)\vee (p'_2 \wedge q'_2)\vee(p_2\wedge q_1)\vee (p'_2 \wedge q'_1)) = \\ (p_1\wedge q_1\wedge  p_1\wedge q_2)\vee (p_1\wedge q_1\wedge  p'_1 \wedge q'_2)\vee(p_1\wedge q_1\wedge  p_2\wedge q_2)\vee (p_1\wedge q_1\wedge  p'_2 \wedge q'_2)\vee \\(p_1\wedge q_1\wedge  p_2\wedge q_1)\vee (p_1\wedge q_1\wedge  p'_2 \wedge q'_1) \vee (p'_1\wedge q'_1\wedge  p_1\wedge q_2)\vee (p'_1\wedge q'_1\wedge  p'_1 \wedge q'_2)\vee \\ (p'_1\wedge q'_1\wedge  p_2\wedge q_2)\vee (p'_1\wedge q'_1\wedge  p'_2 \wedge q'_2)\vee(p'_1\wedge q'_1\wedge  p_2\wedge q_1)\vee (p'_1\wedge q'_1\wedge  p'_2 \wedge q'_1)= \\ (p_1\wedge q_1 \wedge q_2)\vee 0\vee(p_1\wedge q_1\wedge  p_2\wedge q_2)\vee (p_1\wedge q_1\wedge  p'_2 \wedge q'_2)\vee \\(p_1\wedge q_1\wedge  p_2)\vee 0 \vee 0\vee (p'_1\wedge q'_1 \wedge q'_2)\vee \\ (p'_1\wedge q'_1\wedge  p_2\wedge q_2)\vee (p'_1\wedge q'_1\wedge  p'_2 \wedge q'_2)\vee 0\vee (p'_1\wedge q'_1\wedge  p'_2 )= \\ (p_1\wedge q_1 \wedge q_2)\vee(p_1\wedge q_1\wedge  p_2\wedge q_2)\vee (p_1\wedge q_1\wedge  p'_2 \wedge q'_2)\vee (p_1\wedge q_1\wedge  p_2) \\ \vee (p'_1\wedge q'_1 \wedge q'_2)\vee (p'_1\wedge q'_1\wedge  p_2\wedge q_2)\vee (p'_1\wedge q'_1\wedge  p'_2 \wedge q'_2)\vee (p'_1\wedge q'_1\wedge  p'_2 )= \\(p_1\wedge((q_1 \wedge q_2)\vee( q_1\wedge  p_2\wedge q_2)\vee (q_1\wedge  p'_2 \wedge q'_2)\vee ( q_1\wedge  p_2)))\vee \\ (p_1'\wedge((q'_1 \wedge q'_2)\vee (q'_1\wedge  p_2\wedge q_2)\vee (q'_1\wedge  p'_2 \wedge q'_2)\vee (q'_1\wedge  p'_2 )))= \\ (p_1\wedge((q_1 \wedge q_2)\vee (q_1\wedge(( p'_2 \wedge q'_2)\vee p_2))))\vee  (p_1'\wedge((q'_1 \wedge q'_2)\vee (q'_1\wedge ((p_2\wedge q_2)\vee p'_2) )))= \\ (p_1\wedge q_1\wedge (q_2\vee ( p'_2 \wedge q'_2)\vee p_2))\vee (p_1'\wedge q'_1 \wedge (q'_2 \vee (p_2\wedge q_2) \vee p'_2) )=  \\ (p_1\wedge q_1\wedge ((q_2 \vee p_2) \vee (q_2 \vee p_2)'))\vee (p_1'\wedge q'_1 \wedge ((p_2\wedge q_2)'\vee (p_2\wedge q_2)) )= \\ (p_1\wedge q_1\wedge 1)\vee (p_1'\wedge q'_1 \wedge 1)= (p_1\wedge q_1)\vee (p_1'\wedge q'_1)=f(p_1,q_1)
\end{multline}
from which we conclude 
\begin{equation}
f(p_1,q_1)\leq f(p_1,q_2)\vee f(p_2,q_2)\vee f(p_2,q_1). 
\end{equation}
Following Jaynes \cite{Jaynes2003} we may assign to every proposition $p$ a degree of plausibility $P(p)\in\mathbb{R}$. Every sensible way of assigning such degrees of plausability must be such that if $p\leq q$ then $P(p)\leq P(q)$. Therefore we find what we will call Bell's inequalities
\begin{equation}
P(f(p_1,q_1))\leq P(f(p_1,q_2)\vee f(p_2,q_2)\vee f(p_2,q_1)). 
\end{equation}

Now, given that quantum mechanics seems to correctly predict the behaviour of light polarization, we may go back to out previous example and test Bell's inequalities.
\begin{example}
Suppose $P_A(\theta)=P(\theta)\otimes id_{\mathbb{C}^2}$ and $P_B(\theta)=id_{\mathbb{C}^2}\otimes P(\theta)$. This means that $P_A(\theta)$ measures on the first photon and $P_B(\theta)$ on the second. More precisely we may interpret the expected value of $P_A(\theta)$ as yielding a degree of plausability to the proposition $p(\theta)=$"\textit{the first photon has linear polarization at an angle $\theta$}" and $P_B(\theta)$ playing the analogue role for $q(\theta)=$"\textit{the second photon has linear polarization at an angle $\theta$}". Therefore we find that the degree of plausibility for the proposition $p\wedge q$ is
\begin{multline}
tr(P_A(\alpha)P_B(\beta)\rho_\psi)= \langle \psi, P_A(\alpha)P_B(\beta)\psi\rangle= \\ \frac{1}{\sqrt{2}}\langle \psi, P(\alpha)|0\rangle\otimes P(\beta)|\pi/2\rangle -P(\alpha)|\pi/2\rangle\otimes P(\beta)|0\rangle\rangle= \\ \frac{1}{\sqrt{2}}\langle \psi, \cos(\alpha)|\alpha\rangle\otimes\sin(\beta)|\beta\rangle -\sin(\alpha)|\alpha\rangle\otimes\cos(\beta)|\beta\rangle\rangle= \\ \frac{1}{2}\left(\cos^2(\alpha)\sin^2(\beta)-2\cos(\alpha)\sin(\alpha)\sin(\beta)\cos(\beta))+\sin^2(\alpha)\cos^2(\beta)\right) = \\ \frac{1}{2}\left(\cos(\alpha)\sin(\beta)-\sin(\alpha)\cos(\beta)\right)^2 = \frac{1}{2}\sin^2(\alpha-\beta).
\end{multline}  				
Since the degree of plausability of $p(\alpha)'q(\beta)'$ is given by adding $\pi/2$ to both angles we find that $P(f(p(\alpha),q(\beta)))=sin^2(\alpha-\beta)$. Therefore we obtain through Bell's inequalities
\begin{multline}
1=sin^2(0-\pi/2)= P(f(p(0),q(\pi/2)))\leq \\ P(f(p(0),q(\pi/6)))+P(f(p(\pi/3),q(\pi/6)))+P(f(p(\pi/3),q(\pi/2))) = \\ sin^2(0-\pi/6) + sin^2(\pi/3-\pi/6) + sin^2(\pi/3-\pi/2) = 3/4
\end{multline}
which is clealy a contradiction.
\end{example}
We find thus through the contradiction between Bell's inequalities and experiment that we failed in our search of a complete theory of physics according to the definitions given by EPR. In his paper \cite{Bell1964}, Bell found his inequalities by assuming there was a hidden probability space (as in the classical case) from which we could assign degrees of plausibility to propositions. Of course such a view point falls within our discussion and makes it clear that there are no hidden variables. Nonetheless our exposition shows that the problem with the critique to quantum mechanics made by EPR lies in their definitions. Bell's inequalities show that no theory satisfying their requirements for completeness (which we interpreted as having a boolean logical structure) will ever be found. Moreover, our discussion yielded a clearer view on the root of the distinction between quantum mechanics and previous theories: \textit{the logical structure}.

\section{Quantum Logic}

To study the logical structure of quantum mechanics we first discuss the notion of proposition in the theory. Since propositions have to be observables with two possible outcomes "true" or "false", we identify the with the self-adjoint whose spectrum is $\{0,1\}$. These are precisely the orthogonal projections on a Hilbert space $\mathcal{H}$.

\begin{theorem}
Every closed subspace of $\mathcal{H}$ is the image of an orthogonal projection. Conversely, the image of every orthogonal projection is closed.
\end{theorem}
\begin{proof}
Let $V\subseteq\mathcal{H}$ be a closed subspace.  By the Orthogonal Decomposition Theorem we have $\mathcal{H}=V\bigoplus V^{\bot}$. Therefore take the orthogonal projection $\psi\mapsto\xi$ where $\xi$ is the unique element of $V$ such that there exists a $\zeta\in V^{\bot}$ such that $\psi=\xi+\zeta$.
Let $P$ be an orthogonal projection. Then $P(\mathcal{H})=(\ker P)^{\bot}$ and, since every orthogonal complement is closed, $P(\mathcal{H})$ is closed.
\end{proof}

Therefore we see that the set of propositions can also be identified with the closed subspaces of $\mathcal{H}$. From now on we won't make a distintion between quantum propostions, orthogonal projections and closed subspaces and we will denote such an identification by $L(\mathcal{H})$. Both of these identifications will help us endow the quantum mechanical propositions with a logical structure.

\begin{theorem}
The set of closed subspaces of a Hilbert space $\mathcal{H}$ is naturally a poset when equipped with the relation of set inclusion. This is of bounded by $\{0\}$ and $\mathcal{H}$. Moreover, it is a lattice where for every family of closed subspaces $\mathcal{C}$ we have $\bigwedge \mathcal{C} = \bigcap \mathcal{C}$ and $\bigvee \mathcal{C} = \overline{\text{span}\left(\bigcup\mathcal{C}\right)}$.
\end{theorem}

\begin{proof}
Note that if $X$ is a set then $(P(X),\subseteq)$ is a poset and for every $A\subseteq P(X)$ it remains true that $(A,\subseteq)$ is a poset. The case of closed subspaces of a Hilbert space is a special case of this. Moreover, recall that in $(P(X),\subseteq)$ we have for $\mathcal{A}\in X$ that $\bigwedge A = \bigcap \mathcal{A}$ and since intersection of closed subspaces is a closed subspace, this remains true for our case of interest. Similarly $\bigvee\mathcal{A}=\bigcup\mathcal{A}$. But in general the union of subspaces is not a subspace. Nevertheless, the smallest subspace that contains a subset is its span. But we may still run into trouble because the span may not be closed. We can solve this by noticing that the smallest closed set that contains a subset is its closure, yielding the formula for the join in the theorem. Finally it is clear that application of the formulae for the meet and join yield $0=\{0\}$ and $1=\mathcal{H}$. 
\end{proof}

Notice now that in general a closed subspace of $\mathcal{H}$ has many different complements. For example in $\mathcal{C}^2$ we have that $\text{span}(\{(\cos(\theta),\sin(\theta))\})$ is a complement of $\text{span}(\{(1,0)\})$ for all $\theta\in (0,\pi)$. Therefore by theorem \ref{thm:distributive}{thm:distributive} the lattice of quantum propositions cannot be boolean. This explains the root of the contradiction in Bell's inequalities as well as the EPR paradox.

Finally, we would like to make use of the logical structure of quantum mechanics to explain the objects appearing in section \ref{sec:QM}. First of all, notice that the operator $P_A(E)$ corresponding to an observable $A$ and a Borel set $E$ is the orthogonal projection corresponding to the proposition "\textit{measurement of the observable $A$ yields a value in the Borel set $E$}." Secondly, a reasonable way to define a state in quantum mechanics would be as a mapping that assigned to every proposition a degree of plausibility. Precisely,

\begin{definition}
A probability measure on the lattice of propositions $L(\mathcal{H})$ on a Hilbert space $\mathcal{H}$ is a map $\mu:L(\mathcal{H})\to [0,1]$ such that $\mu(H)=1$ and for every sequence $(P_n)$ of pairwise orthogonal projections we have $\mu(\bigoplus_{n=0}^{\infty}P_n)=\sum_{n=0}^{\infty}\mu(P_n)$. 
\end{definition} 

To relate the definition above to the one we gave in section \ref{sec:QM} we may note that for every density operator $\rho$ on a Hilbert space $\mathcal{H}$ the function $\mu_\rho:L(\mathcal{H})\to [0,1]:P\mapsto tr(P\rho)$ is a probability measure on $L(\mathcal{H})$. Conversely,

\begin{theorem}[Gleason's Theorem]
If $\mathcal{H}$ is a Hilbert space with dimension greater than $2$ then every probability measure on $L(\mathcal{H})$ is of the form $\mu_\rho$ for some density operator $\rho$ on $\mathcal{H}$.
\end{theorem}

\chapter{Algebraic Quantum Physics}\label{chp:algebra}
Now that we've understood classical and quantum mechanics as probability theories and displayed their differences, we will now concern ourselves with the development of algebraic methods that will allow us to describe both classical and quantum mechanics in the same framework and to discuss equilibrium further.
 
\section{C*-algebras}

We will start by getting acquainted with the notion of a C*-algebra. This is the mathematical structure we will endow our physical observables with. Even though the general need for this structure can be inspired by the abstract analysis of experimental apparatuses\cite{Strocchi2008a} we will instead give the abstract definition and then justify it through examples. 

\begin{definition}
An algebra $\mathcal{A}$ is a set equipped with three operations:
\begin{align}
\begin{split}
 \mathcal{A} \times \mathcal{A} & \rightarrow  \mathcal{A} \\
 (x,y) & \mapsto  x+y \quad\text{addition;}  \\
 \mathbb{C} \times \mathcal{A} & \rightarrow \mathcal{A} \\
 (\lambda,x) & \mapsto \lambda x \quad\text{scalar multiplication;} \\
 \mathcal{A} \times \mathcal{A} & \rightarrow \mathcal{A} \\
 (x, y) & \mapsto xy \quad\text{multiplication;}
\end{split}
\end{align}
such that with addition and scalar multiplication it forms a complex vector space, with addition and multiplication it forms a ring, and there is a compatibility condition between scalar multiplication and multiplication which is that for all $x,y\in\mathcal{A}$ and $\lambda\in\mathbb{C}$ we have $(\lambda x)y=x(\lambda y) = \lambda (xy)$. If the ring is commutative the algebra is said to be commutative and if the ring is unital the algebra is said to be unital. A norm on an algebra $\mathcal{A}$ is a norm on the vector space structure $\|\cdot\|:\mathcal{A} \rightarrow \mathbb{R}^{+}_0 $ such that for all $x,y\in\mathcal{A}$ we have $\|xy\|\leq\|x\|\|y\|$. An algebra endowed with a norm is called a normed algebra. If the normed vector space structure of an algebra is Banach, the algebra is called Banach. An involution on an algebra $\mathcal{A}$ is a map $^*:\mathcal{A}  \rightarrow \mathcal{A} \quad x \mapsto x^*$ such that for all $x,y\in\mathcal{A}$ and $\lambda\in\mathbb{C}$:
\begin{align}
\begin{split}
(\lambda x + y)^*&=\bar{\lambda}x^* + y^*; \\
(xy)^*&=y^*x^*;\\
(x^*)^*&=x.
\end{split}
\end{align}
An algebra equipped with an involution is said to be a *-algebra. A $C*$-algebra is a Banach *-algebra where for all $x\in\mathcal{A}$ 
\begin{equation}
\|x^*x\|=\|x\|^2.
\end{equation}
\end{definition}

\begin{example}
The set of continuous functions vanishing at infinity on a locally compact Hausdorff space $X$, that is the set $C_0(X)$ of continuous $f:X\rightarrow \mathbb{C}$ such that for every $\epsilon\in\mathbb{R}^+$ there exists a compact set $K$ such that $f(K^c)\subseteq B(0,\epsilon) \subseteq \mathbb{C}$ forms a $C^*$-algebra with the supremum norm
\begin{equation}
\|f\| = \sup\{|f(x)||x\in X\}.
\end{equation}
This algebra differs from the structure described in \ref{sec:classical_probability} in that the functions are admittedly complex and their behavior at infinity is restricted. Nevertheless $C_0(X)$ is unital if and only if $X$ is compact. In that case $C_0(X)=C(X)$ and the observables coincide with the self-adjoint elements of the $C^*$-algebra. One can associate both the need for restricting behavior at infinity or making the space compact by noting that any real feasible experiment performed on a system should be localized. This has to do with the experimental motivation of $C^*$-algebras given in \cite{Strocchi2008a}.
We now note that every commutative $C^*$-algebra can be realized as the space of continuous functions on a compact Hausdorff space. 
\end{example}

\begin{example}
The set of bounded operators in a Hilbert space $\mathcal{H}$ forms a $C^*$-algebra with the operator norm
\begin{equation}
\|A\|=\sup\left\{\left.\frac{\|Ax\|}{\|x\|}\right|x\in\mathcal{H}\setminus \{0\} \right\}.
\end{equation}
Once again, this algebra differs from the structure given in section \ref{sec:QM} because we only consider bounded operators. Once again at a fundamental level this doesn't matter since we know through the spectral theorem or the logical structure presented in section \ref{sec:Q_logic} that we can describe all observables (bounded or unbounded) through their spectral decomposition into projections. In particular, we should be able to take the $C^*$-algebra generated by the projections associated to the observable we want to analyze. For example, instead of considering the position operator $q$ on $L^2(\mathcal{H})$ given by $q\psi(x)=x\psi(x)$ for all $\psi\in\mathcal{H}$, we can consider the $C^*$-algebra generated by the characteristic functions of Borel sets $E\subseteq\mathbb{R}$ whose action on the Hilbert space is $\chi_E\psi(x) = \chi_E(x)\psi(x)$. Moreover, this problem as in the classical case, this problem is related to the fact that no experimental apparatus has an infinite display of outcomes. One indeed cannot measure infinitely large positions or momenta.

Another solution for the case of Schrödinger's mechanics is to consider the Weyl operators $U(a)$ and $U(b)$ for $a,b\in\mathbb{R}$ given by
\begin{align}
\begin{split}
U(a)\psi(x) &= \psi(x-\hbar a) \\
V(b)\psi(x) &= e^{-ibx}\psi(x).
\end{split} 
\end{align}
By Stone's theorem if $q$ is the position operator and $p$ is the momentum operator satisfying the canonical commutation relations $[x,p]=i\hbar$ we have $U(a)=e^{-iap}$ and $V(b)=e^{-ibq}$\cite{Strocchi2008a}. 
\end{example}

As mentioned before the above definition gives structure to the observables of a system. To get a complete kinematical description we need to also give structure to the notion of state. We can inspire the definition of a state by the fact that both in classical and quantum descriptions the statistically appropriate notion of state seemed to act on the observable either through equation \ref{eq:classical_states} or \ref{eq:quantum_states}.

\begin{definition}
A state on a $C*$-algebra $\mathcal{A}$ is a positive normalized linear functional $\omega:\mathcal{A}\rightarrow \mathbb{C}$, i.e. it is a linear map such that $\|\omega\| = 1$ (normalized) and for all $x\in\mathcal{A}$ we have $\omega(x^*x)\geq 0$ (positive).
\end{definition}  

Note that for a unital $C^*$-algebra a positive linear functional $\omega$ is normalized if and only if $\omega(1)=1$.

\begin{example}
By Riesz's representation theorem \cite{Hewitt1975} we have that for every state $\omega$ on $C(X)$ for $X$ compact Hausdorff there exists a probability measure $P$ on $X$ such that
\begin{equation}
\omega (f)=\int fdP
\end{equation}    
for every $f$ in $C(X)$. Indeed $P$ is the measure induced by the Daniel extension of $\omega$. Moreover, it turns out that every commutative $C^*$-algebra is isomorphic to an algebra $C_0(X)$ for $X$ locally compact Hausdorff\cite{Bratteli1997}. In particular, this final remark justifies that classical systems can be treated in the context of $C^*$-algebras. 
\end{example}

\begin{example}
As in the classical case, it so happens that every $C^*-algebra$ can be realized as a closed self-adjoint subalgebra of the algebra of bounded operators of a Hilbert space\cite{Bratteli1997}. This justifies the connection between $C^*$-algebras and the usual presentation of quantum mechanics given in \ref{sec:QM}.
\end{example}

\section{GNS Construction}

Although we won't prove the structure theorems mentioned above for the characterization of $C^*$-algebras we will indeed be interested in the representation of a $C^*$-algebra on a Hilbert space induced by a state. 

\begin{definition}
A representation of a a $C^*$-algebra $\mathcal{A}$ is a tuple $(\mathcal{H},\pi)$ where $\mathcal{H}$ is a Hilbert space and $\pi:\mathcal{A}\rightarrow L(\mathcal{H})$ is a *-homomorphism (i.e. an adjoint preserving homomorphism). If $\mathcal{H}$ has non trivial invariant subspaces under the action of $\pi(\mathcal{A})$ then the representation is said to be reducible.
\end{definition}

\begin{theorem}
If $\mathcal{A}$ is a unital $C^*$-algebra and $\omega$ is a state on it, then there exists a representation $(\mathcal{H}_\omega,\pi_\omega)$ with a unit vector $\Omega_\omega$ such that $\overline{\pi_\omega(\mathcal{A})\Omega_\omega}=\mathcal{H}_\omega$ (i.e., $\Omega_\omega$ is cyclic) and for all $x\in\mathcal{A}$ we have that $\omega(x)=\langle\Omega_\omega,\pi_\omega(x)\Omega_\omega\rangle=tr(\pi_\omega(x)\rho_{\Omega_\omega})$.  
\end{theorem}

\begin{proof}
Notice that in particular $\mathcal{A}$ is a vector space. Consider the function
\begin{align}
\begin{split}
\mathcal{A}\times\mathcal{A}&\rightarrow\mathbb{C} \\
(x,y)&\mapsto\omega(x^*y).
\end{split}
\end{align}
One can show that this function is an inner product except for the fact that there may be elements $x\in\mathcal{H}\setminus\{0\}$ such that $\omega(x^*x)=0$. We may define $\mathcal{N}_\omega := \{x\in\mathcal{A}|\omega(x^*x)=0\}$. Notice that if $x\in\mathcal{N}_\omega$ and $y\in\mathcal{A}$ then 
\begin{align}
\begin{split}
|\omega((yx)^*(yx))|^2&=|\omega(x^*y^*yx)|^2=|\omega((y^*yx)^*x)|^2 \\
&\leq\omega((y^*yx)^*(y^*yx))\omega(x^*x)=0,
\end{split}
\end{align}
that is, $\mathcal{N}_\omega$ is a left ideal of $\mathcal{A}$. Notice that now the inner product
\begin{align}
\begin{split}
\mathcal{A}/\mathcal{N}_\omega\times\mathcal{A}/\mathcal{N}_\omega&\rightarrow\mathbb{C} \\
([x],[y])&\mapsto\langle[x],[y]\rangle:=\omega(x^*y)
\end{split}
\end{align}
is well defined and therefore take $\mathcal{H}_\omega=\overline{\mathcal{A}/\mathcal{N}_\omega}$. We define 
\begin{align}
\begin{split}
\pi_\omega:\mathcal{A}&\rightarrow L(\mathcal{H}_\omega) \\
x&\mapsto \pi_\omega(x)
\end{split}
\end{align}
by extension of $\pi_\omega(x)[y]:=[xy]$ on $\mathcal{A}/\mathcal{N}_\omega$. We define at last $\Omega_\omega:=[1]$. If $x\in\mathcal{A}$ we have
\begin{equation}\label{eqn:state_representation}
\langle \Omega_\omega, \pi_\omega(x)\Omega_\omega\rangle = \langle \Omega_\omega, [x]\rangle = \omega(x). 
\end{equation}
Moreover $\pi_\omega(\mathcal{A})\Omega_\omega = \mathcal{A}/\mathcal{N}_\omega$ and it is therefore verified that the vector $\Omega_\omega$ is cyclic.  
\end{proof}

\begin{example}\label{example:M2}
Let's follow the GNS construction with the example of the $C^*$-algebra of $2\times 2$ matrices with complex entries $M_2(\mathbb{C})$. This is of physical importance for 2 state systems. For example our recurring system in example \ref{ex:Bell} has this algebra of observables (the canonical matrix representations of the operators $P(0)$, $P(\pi/4)$, $P(\pi/2)$ and $P(3\pi/4)$ generate this algebra). Let the elementary matrices of $M_2(\mathbb{C})$ be $E_{ij} = ((\delta_{in}\delta_{jm})_{nm})$. Let's choose the state
\begin{equation}
\omega_\lambda(\alpha) = \lambda \alpha_{11} + (1-\lambda)\alpha_{22}
\end{equation} 
for some $\lambda\in [0,1]$. The parameter $\lambda$ can be given interpretation by noting that $\omega_\lambda(P(0))=\lambda$, that is, $\lambda$ is the expectation value of the photon described to have polarization along the horizontal axis. We have that
\begin{align}
\begin{split}
\omega_\lambda (\alpha^*\alpha) & = \omega_\lambda\left(\left(\sum_{i=1}^2 (\alpha^*)_{ik}\alpha_{kj}\right)_{ij}\right) = \omega_\lambda\left(\left(\sum_{i=1}^2 \overline{\alpha}_{ki}\alpha_{kj}\right)_{ij}\right) \\
& = \lambda(|\alpha_{11}|^2+|\alpha_{21}|^2) + (1-\lambda)(|\alpha_{12}|^2+|\alpha_{22}|^2).
\end{split}
\end{align}
Therefore the ideal $\mathcal{N}_\lambda := \mathcal{N}_{\omega_\lambda}$ will depend on the choice of $\lambda$.
\begin{itemize}
\item If $\lambda = 0$, 
\begin{equation}
\mathcal{N}_0 = \{\alpha\in M_2(\mathbb{C})|\alpha_{12}=\alpha_{22}=0\}.
\end{equation}
Therefore it is clear that if $\mathcal{H}_\lambda:=\mathcal{H}_{\omega_\lambda}$ we have
\begin{equation}
\mathcal{H}_0=M_2(\mathbb{C})/\mathcal{N}_0\simeq\left\{\left.\begin{bmatrix}
0 & \alpha_{12} \\
0 & \alpha_{22}
\end{bmatrix}\right|\alpha_{12},\alpha_{22}\in\mathbb{C}\right\}.
\end{equation}
\item If $\lambda = 1 $ we have the symmetric case and we conclude
\begin{equation}
\mathcal{H}_1=M_2(\mathbb{C})/\mathcal{N}_1\simeq\left\{\left.\begin{bmatrix}
\alpha_{11} & 0 \\
\alpha_{21} & 0
\end{bmatrix}\right|\alpha_{11},\alpha_{21}\in\mathbb{C}\right\}.
\end{equation}
\item If $\lambda \in(0,1)$ we have that $\mathcal{N}_\lambda = \{0\}$ and therefore $M_2(\mathbb{C})/\mathcal{N}_\lambda \simeq M_2(\mathbb{C})$. We have in particular that this representation can be decomposed into the two previous representations
\begin{equation}
M_2(\mathbb{C})=\left\{\left.\begin{bmatrix}
\alpha_{11} & 0 \\
\alpha_{21} & 0
\end{bmatrix}\right|\alpha_{11},\alpha_{21}\in\mathbb{C}\right\}\oplus\left\{\left.\begin{bmatrix}
0 & \alpha_{12} \\
0 & \alpha_{22}
\end{bmatrix}\right|\alpha_{12},\alpha_{22}\in\mathbb{C}\right\}.
\end{equation} 
Moreover, if $\alpha \in M_2(\mathbb{C})$ we have
\begin{equation}
(\pi_{\Omega_{\omega_\lambda}}(\alpha)E_{ij})_{nm}=\sum_{k=1}^2\alpha_{nk}\delta_{ik}\delta_{jm}=\alpha_{ni}\delta_{jm}
\end{equation}
and therefore the spaces in the decomposition are invariant under the action of the representation of the algebra. In particular, we check that the projection $\rho_{\Omega_{\omega_\lambda}}$ onto $\Omega_{\omega_\lambda}$ cannot be of the form $\pi_{\Omega_{\omega_\lambda}}(\alpha)$ for some $\alpha\in M_2(\mathbb{C})$ since it doesn't respect that invariance
\begin{equation}
\rho_{\Omega_{\omega_\lambda}}(E_{ij})=\langle \Omega_{\omega_\lambda}, E_{ij}\rangle\Omega_{\omega_\lambda} = \omega_\lambda(E_{ij})I_2=(\lambda\delta_{1i}\delta_{1j}+(1-\lambda)\delta_{2i}\delta_{2j})I_2.
\end{equation} 
\end{itemize}
\end{example}

Given equation \ref{eqn:state_representation} one may feel tempted to associate to the system the orthogonal projection $\rho_{\Omega_{\omega_\lambda}}$ onto $\Omega_{\omega_\lambda}$ as a state. This would yield according to equation \ref{eqn:entropy_pure} a state of zero entropy. We need to find a way around this. Now, examining our previous example where the state was conveniently written as a convex sum of states, we find that the extremal points of this sum (the cases $\lambda\in\{0,1\}$) generate irreducible representations of the algebra while the other cases didn't. Moreover, the actual state $\rho_{\Omega_{\omega_\lambda}}$ was not in the image of the algebra of observables in the reducible representations considered. This inspires us to try to find a state $\rho_{\omega_\lambda}$ which also satisfies $tr(\pi_{\omega_lambda}(\alpha)\rho_{\omega_\lambda})=\omega(\alpha)$ from the irreducible representations in the GNS construction.

Being concerned for the moment with finite dimensional representations, we will in general be able to write
\begin{equation}
\mathcal{H}_\omega = \bigoplus_{\beta\in I}\mathcal{H}_\omega^{(\beta)}
\end{equation}
where $\{\mathcal{H}_\omega^{(\beta)}|\beta\in I\}$ is a set of irreducible representations of $\mathcal{A}$. The decomposition leaves the projection operators $P^{(\beta)}$ onto $\mathcal{H}_\omega^{(\beta)}$ such that
\begin{equation}
id_{\mathcal{H}_\omega}=\sum_{\beta\in I}P^{(\beta)}.
\end{equation}
Therefore, we have if $\{e_1,\cdots,e_n\}$ is a basis for $\mathcal{H}_\omega$
\begin{align}
\begin{split}
\omega(\alpha)&=\langle \Omega_\omega, \pi_\omega(\alpha)\Omega_\omega\rangle \\
&=\langle\Omega_\omega,\sum_{\beta\in I}P^{(\beta)}\pi_\omega(\alpha)\Omega_\omega\rangle \\
&=\langle\Omega_\omega,\sum_{\beta\in I}P^{(\beta)}\pi_\omega(\alpha)P^{(\beta)}\Omega_\omega\rangle \\
&=\langle\Omega_\omega,\sum_{m=1}^n\langle e_m,\sum_{\beta\in I}P^{(\beta)}\pi_\omega(\alpha) P^{(\beta)}\Omega_\omega\rangle e_m\rangle \\
&=\sum_{m=1}^n\langle e_m,\sum_{\beta\in I}P^{(\beta)}\pi_\omega(\alpha) P^{(\beta)}\langle\Omega_\omega, e_m\rangle\Omega_\omega\rangle \\
&=\sum_{m=1}^n\langle e_m,\sum_{\beta\in I}P^{(\beta)}\pi_\omega(\alpha) P^{(\beta)}\rho_{\Omega_\omega} e_m\rangle \\
&=tr(\sum_{\beta\in I}P^{(\beta)}\pi_\omega(\alpha) P^{(\beta)}\rho_{\Omega_\omega})\\
&=tr(\pi_\omega(\alpha)\sum_{\beta\in I}P^{(\beta)}\rho_{\Omega_\omega} P^{(\beta)}).
\end{split}
\end{align}
Therefore we define
\begin{equation}
\rho_\omega := \sum_{\beta\in I}P^{(\beta)}\rho_{\Omega_\omega} P^{(\beta)}
\end{equation}
as the induced state.

\begin{example}
Continuing with example \ref{example:M2} we find that since in the cases $\lambda\in\{0,1\}$ since the representation is irreducible we have $\rho_{\omega_\lambda}=\rho_{\Omega_{\omega_\lambda}}$ and therefore the state is pure and has null entropy. In the case $\lambda\in (0,1)$ we have for all $\alpha\in M_2(\mathbb{C})$
\begin{align}
\begin{split}
\rho_{\omega_\lambda}\alpha &= \sum_{i=1}^2 P^{(i)}\rho_{\Omega_{\omega_\lambda}}P^{(i)}\alpha \\
& = P^{(1)}\rho_{\Omega_{\omega_\lambda}}\begin{bmatrix}
\alpha_{11} & 0 \\
\alpha_{21} & 0 
\end{bmatrix} + P^{(2)}\rho_{\Omega_{\omega_\lambda}}\begin{bmatrix}
0 & \alpha_{12} \\
0 & \alpha_{22} 
\end{bmatrix} \\
& = P^{(1)}\omega\left(\begin{bmatrix}
\alpha_{11} & 0 \\
\alpha_{21} & 0 
\end{bmatrix} \right)I_2 + P^{(2)}\omega\left(\begin{bmatrix}
0 & \alpha_{12} \\
0 & \alpha_{22} 
\end{bmatrix} \right)I_2 \\
&= P^{(1)}\lambda\alpha_{11}I_2 + P^{(2)}(1-\lambda)\alpha_{22}I_2 =\lambda\alpha_{11} E_{11} + (1-\lambda)\alpha_{22}E_{22} \\
& = \lambda\rho_{E_{11}}\alpha + (1-\lambda)\rho_{E_{22}}\alpha = (\lambda\rho_{E_{11}} + (1-\lambda)\rho_{E_{22}})\alpha
\end{split}
\end{align} 
and therefore $\rho_{\omega_\lambda}=\lambda\rho_{E_{11}} + (1-\lambda)\rho_{E_{22}}$.
We conclude that the entropy is 
\begin{equation}\label{eq:entropy_M2}
S = -(\lambda\log (\lambda) + (1-\lambda)\log(1-\lambda))
\end{equation}
\begin{figure}
\centering
\includegraphics[width=0.8\textwidth]{entropia_M2.png}
\caption{The entropy of equation \ref{eq:entropy_M2} as a function of the probability that the photon has horizontal polarization.}
\end{figure}
\end{example}

\chapter{KMS States}\label{chp:KMS}
Having developed the theory of algebraic quantum mechanics we are now in the correct setting to discuss the theory of KMS states as shown in \cite{Duvenhage1999}. Although starting with an abstract definition, we will use the case of finite dimensional quantum systems to inspire why this is a natural generalization of the Gibbs states presented in \ref{ex:Gibbs}. In particular, they will be invariant under the dynamics of the system justifying therefore their usefulness given the generality of the dynamics we've defined.

\section{Definition and Dynamical Invariance}

\begin{definition}\label{def:KMS}
Let $(\mathcal{A},\tau)$ be a $C^*$ or $W^*$-dynamical system, $\omega$ a state on $\mathcal{A}$ (in the $W^*$ case we demand $\omega$ is normal), $\beta\in\mathbb{R}$,
\begin{align}
\begin{split}
\mathfrak{D}_\beta=\left\{
\begin{array}{lr}
\{z\in\mathbb{C}|0<\im z<\beta\} &\beta \geq 0 \\
\{z\in\mathbb{C}|\beta<\im z<0\} &\beta < 0
\end{array}\right.,
\end{split}
\end{align}  
and $\overline{\mathfrak{D}_\beta}$ be the closure $\mathfrak{D}_\beta$ except for the case $\beta=0$ where we set $\overline{\mathfrak{D}_\beta}=\mathbb{R}$ (we will keep using these sets during the rest of this work). $\omega$ is said to be a $(\tau,\beta)$-KMS state if it satisfies the KMS conditions, that is, for every $A,B\in\mathcal{A}$ there exists a bounded continuous function $F_{A,B}:\overline{\mathfrak{D}_\beta}\rightarrow \mathbb{C}$ (which we will usually refer to as a witness to $\omega$ being a $(\tau,\beta)$-KMS state) analytic on $\mathfrak{D}_\beta$ and such that for every $t\in\mathbb{R}$ it is true that
\begin{align}
\begin{split}
F_{A,B}(t)&=\omega(A\tau_t(B)) \\
F_{A,B}(t+i\beta)&=\omega(\tau_t(B)A).
\end{split}
\end{align}
A $(\tau,-1)$-KMS state is called a $\tau$-KMS state.
\end{definition}

Although the definition of $\tau$-KMS state may seem bizarre since it corresponds to negative temperatures, it is of great technical importance. Indeed the next theorem shows that for the most part everything we learn about $\tau$-KMS states is true for $(\tau,\beta)$-KMS states.

\begin{theorem}
Let $(\mathcal{A},\tau)$ be a $C^*(W^*)$-dynamical system, $\omega$ a state on $\mathcal{A}$, and $\beta\in\mathbb{R}$. Define
\begin{alignat}{2}
\alpha:\mathbb{R}&\rightarrow & \Aut(\mathcal{A}) \nonumber \\
t&\mapsto &\alpha_t:\mathcal{A} & \rightarrow\mathcal{A} \\
&&A&\mapsto \alpha_t(A):=\tau_{-\beta t}(A). \nonumber
\end{alignat}
Then $(\mathcal{A},\alpha)$ is a $C^*(W^*)$-dynamical system and:
\begin{itemize}
\item if $\omega$ is a $(\tau,\beta)$-KMS state then it is an $\alpha$-KMS state;
\item if $\beta\neq 0$ then $\omega$ is a $(\tau,\beta)$-KMS state if and only if it is an $\alpha$-KMS state. 
\end{itemize}
\end{theorem}

\begin{proof}
It is easy to see that $(\mathcal{A},\alpha)$ is a $C^*(W^*)$-dynamical system.
If $\omega$ is a $(\tau,\beta)$-KMS state and $F_{A,B}$ is a witness to this then
\begin{align}
\begin{split}
G_{A,B}:\overline{\mathfrak{D}_{-1}}&\rightarrow\mathbb{C} \\
z&\mapsto F_{A,B}(-\beta z)
\end{split}
\end{align} 
clearly shows that it is an $\alpha$-KMS state.
Conversely assume $\beta\neq 0$ and $\omega$ is an $\alpha$-KMS state. Suppose $G_{A,B}$ is a witness to $\omega$ being an $\alpha$-KMS state. Then  
\begin{align}
\begin{split}
F_{A,B}:\overline{\mathfrak{D}_{-1}}&\rightarrow\mathbb{C} \\
z&\mapsto F_{A,B}(-z/\beta)
\end{split}
\end{align} 
clearly shows it is a $(\tau,\beta)$-KMS state.
\end{proof}

The importance of KMS states becomes immediately obvious due to the next theorem since it shows that KMS states are constant in the dynamics.

\begin{theorem}\label{thm:time_invariance}
Let $(\mathcal{A},\tau)$ be a $C^*$ or $W^*$ dynamical system where $A$ is unital and $\omega$ a $(\tau,\beta)$-KMS state for some $\beta\in\mathbb{R}\setminus\{0\}$. Then for all $A\in\mathcal{A}$ and $t\in\mathbb{R}$ we have $\omega(\tau_t(A))=\omega(A)$.
\end{theorem}

\begin{proof}
By the previous theorem, we might as well assume that $\omega$ is a $\tau$-KMS state. Let $A\in\mathcal{A}$ be self-adjoint. If $F_{1,A}$ is a witness to $\omega$ being a $\tau$-KMS state then for $t\in\mathbb{R}$
\begin{equation}
F_{1,A}(t)=\omega(\tau_t(A))=F_{1,A}(t-i)
\end{equation}
and since $\overline{\omega(\tau_t(A))}=\omega(\tau_t(A)^*)=\omega(\tau_t(A^*))=\omega(\tau_t(A))$ we have that $F_{1,A}(\overline{\mathfrak{D}_{-1}}\setminus\mathfrak{D}_{-1})\subseteq\mathbb{R}$. Given that $F_{1,A}$ is continuous, bounded and analytic on $\mathfrak{D}_{-1}$ we conclude that $F_{1,A}$ is constant (this is an application of Liouville's theorem, see \cite{Duvenhage1999}). Therefore the theorem follows for self-adjoint operators.
If $A\in\mathcal{A}$ we have
\begin{equation}
\omega(\tau_t(A))=\omega\left(\tau_t\left(\frac{A+A^*}{2}\right)\right)+i\omega\left(\tau_t\left(\frac{A-A^*}{2i}\right)\right)
\end{equation}
and each of the terms in the sum are independent of $t$ since the operators are self-adjoint. The theorem follows.
\end{proof}

\section{Gibbs states}

Although we've shown that KMS states are constant under the dynamics of a system, this isn't the only requirement for a description of statistical equilibrium. Other issues like stability should be studied. The step we will give into further justifying the study of this states is to show that they are equivalent to the Gibbs states in the case of a finite dimensional Hilbert space. This will be stated with the use of only one theorem inspired by the work in \cite{Duvenhage1999}.

\begin{theorem}
Let $(\mathcal{B}(\mathcal{H}),\tau)$ be the $C^*(W^*)$-dynamical system discussed in example \ref{ex:schrodinger}. Then a state $\omega$ on $\mathcal{B}(\mathcal{H})$ is a $\beta$-Gibbs state if and only if it is a $(\tau,\beta)$-KMS state.  
\end{theorem}

\begin{proof}
Assume $\omega$ is a $\beta$-Gibbs state. Define for $A,B\in\mathcal{B}(\mathcal{H})$
\begin{align}
\begin{split}
F_{A,B}:\mathbb{C}&\rightarrow\mathbb{C} \\
z&\mapsto \omega(A\tau_z(B))
\end{split}
\end{align}
Then, we have for $t\in\mathbb{R}$ that $F_{A,B}(t)=\omega(A\tau_t(B))$ and 
\begin{align}
\begin{split}
F_{A,B}(t+i\beta)& =\omega(A\tau_{t+i\beta}(B))=\frac{\tr(e^{-\beta H}Ae^{iH(t+i\beta)}Be^{-iH(t+i\beta)})}{\tr e^{-\beta H}} \\
&= \frac{\tr(e^{-\beta H}Ae^{iHt}e^{-\beta H}Be^{-iHt}e^{\beta H})}{\tr e^{-\beta H}} \\
&= \frac{\tr(e^{iHt}e^{-\beta H}Be^{-iHt}e^{\beta H}e^{-\beta H}A)}{\tr e^{-\beta H}} \\
&= \frac{\tr(e^{iHt}e^{-\beta H}Be^{-iHt}A)}{\tr e^{-\beta H}} \\
&= \frac{\tr(e^{-\beta H}e^{iHt}Be^{-iHt}A)}{\tr e^{-\beta H}} = \omega(\tau_t(B)A).
\end{split} 
\end{align}
Moreover, $\omega$ is continuous and doesn't depend on $z$, therefore $F_{A,B}$ is analytic (and continuous which easily follows) due to the product rule and the fact that the exponential is analytic. 
If $\{e_1,\dots,e_N\}$ is an orthonormal basis of eigenvectors of $H$ associated to the eigenvalues $E_1,\dots E_N$ and $P_1,\dots P_n$ are the corresponding projections on the span of each of the vectors, we have for $z\in \overline{\mathfrak{D}_\beta}$
\begin{align}
\begin{split}
\|e^{\pm iHz}\|&=\|\sum_{n=1}^N e^{\pm iE_n z}P_n\|\leq\sum_{n=1}^N |e^{\pm iE_n z}|\|P_n\| \\
&=\sum_{n=1}^N |e^{\pm iE_n z}|=\sum_{n=1}^N |e^{\pm iE_n \re z}e^{\mp E_n \im z}|=\sum_{n=1}^N |e^{\mp E_n \im z}| \\
&=\sum_{n=1}^N e^{\mp E_n \im z}\leq \sum_{n=1}^N e^{|E_n\beta|}.
\end{split}
\end{align}  
Since 
\begin{align}
\begin{split}
\|F_{A,B}(z)\|&=\|\omega(A\tau_z(B))\|\leq\|\omega\|\|A\|\|e^{iHz}\|\|B\|\|e^{-iHz}\| \\
&\leq\|A\|\sum_{n=1}^N e^{|E_n\beta|}\|B\|\sum_{n=1}^N e^{|E_n\beta|},
\end{split}
\end{align}
it follows that $F_{A,B}|_{\overline{\mathfrak{D}_\beta}}$ is bounded and that $\omega$ is a $(\tau,\beta)$-KMS state. \\
Now assume that $\omega$ is a $(\tau,\beta)$-KMS state and let $F_{A,B}$ be witness of this for $A,B\in\mathcal{B}(\mathcal{A})$. Define 
\begin{align}
\begin{split}
G_{A,B}:\mathbb{C}&\rightarrow\mathbb{C} \\
z&\mapsto \omega(A\tau_z(B))
\end{split}
\end{align}
We want to show that $F_{A,B}=G_{A,B}|_{\overline{\mathfrak{D}_\beta}}$ or, equivalently, that
\begin{align}
\begin{split}
f:\overline{\mathfrak{D_\beta}}&\rightarrow\mathbb{C} \\
z&\mapsto F_{A,B}(z)-G_{A,B}(z)
\end{split}
\end{align} 
(which is of course continuous and analytic in $\mathfrak{D}_\beta$) is null. In the case $\beta=0$ this is obvious. Assume $\beta>0$. Note that $D=\{z\in\mathbb{C}|-\beta<\im z<\beta\}$ is a region (that is, open and connected) and $D^*=D$. It is clear that $f(\mathbb{R})=\{0\}$. Then, by the Schwarz reflection principle\cite{Conway1978} there exists an analytic function $g:D\rightarrow\mathbb{C}$ such that $g(z)=f(z)$ in $\mathfrak{D}_\beta\cup\mathbb{R}$. Then, since $g(\mathbb{R})={0}$ we have that $g$ is null\cite{Conway1978}. We conclude that $f(z)=0$ for $z\in\mathfrak{D}_\beta\cup\mathbb{R}$ and therefore is null by continuity. The case $\beta<0$ is analogous.\\
In particular we now have that 
\begin{equation}
\omega(A\tau_{i\beta}(B))=G_{A,B}(i\beta)=F_{A,B}(i\beta)=\omega(\tau_0(B)A)=\omega(BA)\footnote{A simple calculation shows that this equation is always true for $\beta$-Gibbs states. In our present situation we need to prove the converse.}.
\end{equation}
Therefore, if $\{e_1,\dots ,e_N\}$ is an orthonormal basis of eigenvectors of $H$ associated to the eigenvalues $E_1,\dots E_N$ and $\{f_1,\dots,f_N\}$ is the dual basis, we have
\begin{align}
\begin{split}
\omega(A)&=\omega\left(\sum_{n,m=1}^N f_n(Ae_m)e_n\otimes e_m\right)=\sum_{n,m=1}^N f_n(Ae_m)\omega(e_n\otimes e_m) \\
&=\frac{1}{\tr(e^{-\beta H})}\sum_{n,m=1}^N f_n(Ae_m)\tr(e^{-\beta H})\omega(e_n \otimes e_m) \\
&=\frac{1}{\tr(e^{-\beta H})}\sum_{n,m,k=1}^N f_n(Ae_m)e^{-\beta E_k}\omega(e_n\otimes e_m) \\
&=\frac{1}{\tr(e^{-\beta H})}\sum_{n,m,k=1}^Nf_n(Ae_m)e^{-\beta E_k}\omega((e_n\otimes e_k)( e_k\otimes e_m)) \\
\end{split}
\end{align}
where we have used that $(e_n\otimes e_k)(e_k\otimes e_m)=\langle e_k, e_k \rangle (e_n\otimes e_m)=e_n\otimes e_m$. Attempting to put the equation in a form where we can apply our previously derived condition we have
\begin{align}
\begin{split}
\omega(A)&=\frac{1}{\tr(e^{-\beta H})}\sum_{n,m,k=1}^Nf_n(Ae_m)e^{-\beta E_m}\omega((e_n\otimes e_k) e^{-\beta E_k} (e_k\otimes e_m) e^{\beta E_m}) \\
&=\frac{1}{\tr(e^{-\beta H})}\sum_{n,m,k=1}^Nf_n(Ae_m)e^{-\beta E_m}\omega((e_n\otimes e_k) e^{-\beta H} (e_k\otimes e_m) e^{\beta H}) \\
&=\frac{1}{\tr(e^{-\beta H})}\sum_{n,m,k=1}^Nf_n(Ae_m)e^{-\beta E_m}\omega((e_n\otimes e_k) \tau_{i\beta} (e_k\otimes e_m)) \\
&=\frac{1}{\tr(e^{-\beta H})}\sum_{n,m,k=1}^Nf_n(Ae_m)e^{-\beta E_m}\omega( (e_k\otimes e_m)( e_n\otimes e_k))\\
&=\frac{1}{\tr(e^{-\beta H})}\sum_{n,m,k=1}^Nf_n(Ae_m)e^{-\beta E_m}\omega( e_k\otimes e_k)\langle e_m,e_n\rangle\\
&=\frac{1}{\tr(e^{-\beta H})}\sum_{n,m=1}^Nf_n(Ae_m)e^{-\beta E_m}\omega\left(\sum_{k=1}^N \rho_{e_k}\right)\delta_{mn}\\
\end{split}
\end{align}
by noticing that $e_k\otimes e_k=\rho_{e_k}$ using the notation presented in section \ref{sec:QM}. Finally, we have that
\begin{align}
\begin{split}
\omega(A)&=\frac{1}{\tr(e^{-\beta H})}\sum_{n,m=1}^Nf_n(Ae_m)e^{-\beta E_m}\omega(1)\delta_{mn} \\
&=\frac{1}{\tr(e^{-\beta H})}\sum_{n=1}^Nf_n(Ae_n)e^{-\beta E_n} =\frac{\tr(Ae^{-\beta H})}{\tr(e^{-\beta H})}
\end{split}
\end{align}
from which follows that $\omega$ is a $\beta$-Gibbs state.
\end{proof}

\chapter{The Modular Theory of Tomita-Takesaki}\label{chp:tomita}
We are now going to develop the theory of Tomita-Takesaki following the approach of \cite{Duvenhage1999} and \cite{Rieffel1977}. This approach is appropriate for our purposes since we want to introduce the main objects of the theory 

\chapter{KMS States and Tomita-Takesaki Theory}\label{chp:final}
Now that we've discussed KMS states and the mathematical tool of Tomita-Takesaki theory we are ready to discuss their relationship. In this chapter we will prove the most important theorem of this work. In particular, we already saw that the only $(\tau,\beta)$-KMS state for Schrodinger's time evolution was the $\beta$-Gibbs state. The theorem we will prove will show that the only possible dynamical law $\tau$ which has the $\beta$-Gibbs state as a $(\tau,\beta)$-KMS state (equilibrium state) is Schrodinger's time evolution. In other words, we provide a derivation of Schrodinger's equation from the axiom of equilibrium described by $\beta$-Gibbs states.  

\chapter{Final Remarks and Further Work}
In this last chapter we will point out some of the topics related to this work that were not addressed. This is in the hopes that an interested reader will start thinking about them and begin working in this beautiful subject.

\begin{itemize}

\item During the first chapters a great emphasis was put on classical mechanics. This was done because the algebraic theory presented is also suitable for the study of classical systems. Nevertheless, KMS states do not seem as an appropriate description of classical thermodynamical equilibrium. Indeed, in the classical case where algebras are commutative the modular group is static: observables do not evolve. Therefore result \ref{thm:final} would yield an undesirable result. Still, works like \cite{Aizenman1977} and \cite{Gallavotti1976} have explored non-trivial classical analogues of the KMS condition and Tomita-Takesaki theory obtaining partial results.

\item During this work we proved two main results which gave us confidence on the use of KMS states for the description of equilibrium. We showed that KMS states where invariant under the dynamical law \ref{thm:time_invariance} and that in the finite dimensional case they where equivalent to Gibbs states \ref{thm:gibbs}. Nevertheless, the arguments in favor of KMS states are multiple. In particular \cite{Haag1992} discusses the appearance of KMS states through the grand canonical ensemble of systems with finite volume (inspired through arguments of ergodicity or the principle of maximal entropy) and through more direct arguments by making use of dynamical stability or passivity. The last two in particular make the case for KMS states without ever considering the thermodynamic limit.

\item The work in \cite{Haag1992} shows the importance of Tomita-Takesaki theory for relativistic quantum theories. Although this theory was used here in the context of statistical physics, it is a very useful tool in the general study and classification of von Neumann algebras. A simple argument puts forth the importance of these structures for relativistic theories. We should be able to assign to regions of spacetime local algebras of observables. The algebras corresponding to causally disconnected regions should commute to ensure the principle of causality. Then the commutant of a local algebra in a region should be intimately connected to the algebra corresponding to the causal complement of the region. 

\item Notice that the definition of a KMS state \ref{def:KMS} seems to distinguish between time and space. This is not the case precisely since the dynamical law does not have to represent time evolution. Nevertheless works like \cite{Bros1994} have extended the KMS condition to the relativistic realm by giving a Lorentz covariant formulation suitable for quantum field theories.

\item Works like  \cite{Reyes2013}, \cite{Balachandran2013c}, and \cite{Balachandran2013} show the vast amount of applications algebraic quantum theory has. We think that of particular importance are the ambiguities in the entropy calculated through the state \ref{eq:entropy} exhibited in \cite{Balachandran2013b} and \cite{Balachandran2013a}.

\end{itemize}


\chapter{Acknowledgements}
\section*{English}

\section*{Español}

\bibliography{/home/ivan/Documents/Bib_Files/Monografia}
\bibliographystyle{ieeetr}

\end{document}
