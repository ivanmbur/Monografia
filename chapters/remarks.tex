In this last chapter we will point out some of the topics related to this work that were not addressed. This is in the hopes that an interested reader will start thinking about them and begin working in this beautiful subject.

\begin{enumerate}[(i)]

\item During the first chapters a great emphasis was put on classical mechanics. This was done because the algebraic theory presented is also suitable for the study of classical systems. Nevertheless, KMS states do not seem as an appropriate description of classical thermodynamical equilibrium. Indeed, in the classical case where algebras are commutative the modular group is static: observables do not evolve. Therefore result \ref{thm:final} would yield an undesirable result. Still, works like \cite{Aizenman1977} and \cite{Gallavotti1976} have explored non-trivial classical analogues of the KMS condition and Tomita-Takesaki theory obtaining partial results.

\item During this work we proved two main results which gave us confidence on the use of KMS states for the description of equilibrium. We showed that KMS states where invariant under the dynamical law \ref{thm:time_invariance} and that in the finite dimensional case they where equivalent to Gibbs states \ref{thm:gibbs}. Nevertheless, the arguments in favor of KMS states are multiple. In particular \cite{Haag1992} discusses the appearance of KMS states through the grand canonical ensemble of systems with finite volume (inspired through arguments of ergodicity or the principle of maximal entropy) and through more direct arguments by making use of dynamical stability or passivity. The last two in particular make the case for KMS states without ever considering the thermodynamic limit.

\item The work in \cite{Haag1992} shows the importance of Tomita-Takesaki theory for relativistic quantum theories. Although this theory was used here in the context of statistical physics, it is a very useful tool in the general study and classification of von Neumann algebras. A simple argument puts forth the importance of these structures for relativistic theories. We should be able to assign to regions of spacetime local algebras of observables. The algebras corresponding to causally disconnected regions should commute to ensure the principle of causality. Then the commutant of a local algebra in a region should be intimately connected to the algebra corresponding to the causal complement of the region. 

\item Notice that the definition of a KMS state \ref{def:KMS} seems to distinguish between time and space. This is not the case precisely since the dynamical law does not have to represent time evolution. Nevertheless works like \cite{Bros1994} have extended the KMS condition to the relativistic realm by giving a Lorentz covariant formulation suitable for quantum field theories.

\item Works like  \cite{Reyes2013}, \cite{Balachandran2013c}, and \cite{Balachandran2013} show the vast amount of applications algebraic quantum theory has. We think that of particular importance are the ambiguities in the entropy calculated through the state \ref{eq:entropy} exhibited in \cite{Balachandran2013b} and \cite{Balachandran2013a}.

\end{enumerate}
