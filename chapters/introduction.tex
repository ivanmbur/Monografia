One of the most important problems in modern physics is that of the mathematical formulation of quantum field theory. Although used successfully throughout the community to study the most relevant problems of particle physics, solid state physics, cosmology, and the dark sector, among others, it still lacks a complete rigorous mathematical formulation. Indeed, divergences and ill-defined symbols rid the whole theory. As physicists, it is not only our duty to give predictions of the physical world but to understand its working principles. In particular, a complete understanding of these must be of a logical nature. It is our believe that mathematics serves this purpose, especially in the cases where those who claim to have an intuitive understanding are often wrong. The purpose of this monograph is to do a bibliographical revision of a result in the realm of mathematical physics and in the search for the correct mathematical framework of quantum theories with infinite degrees of freedom: to every thermodynamical equilibrium quantum state there is a canonical dynamical law governing the time evolution of the system.

In chapter \ref{chp:axiom} we do a quick revision of general frameworks of classical and quantum theories. This serves mainly to establish notation and point out some of the common elements these theories have which will inspire the algebraic approach present throughout this monograph. Before we delve into this unifying scheme, we attempt to understand the differences between these theories in chapter \ref{chp:logic}. We arrive at the conclusion that it has its roots in the mathematical structure of the propositions associated to a quantum theory. Indeed, the structure will not be that of a boolean algebra\footnote{This departure from the mathematical framework of classical propositions is the root of my skepticism towards those who claim to have an intuitive understanding of quantum theory.}. In chapter \ref{chp:algebra} we present the general framework of algebraic quantum theory. This is inspired by the features noted in chapter \ref{chp:axiom}. In particular, we develop the theory of the GNS construction (which we exemplify by showing how it can aid in the calculation of entropies) and of dynamical systems, the two stepping stones in the path to the main result. In chapter \ref{chp:KMS} we study KMS states. These will be states characterized by certain analytic properties (the KMS condition) which we will interpret as those of quantum states in thermodynamical equilibrium. This will be inspired by studying the relationship between KMS states and Gibbs states (the canonical ensemble) in finite dimensional quantum systems. Chapter \ref{chp:tomita} develops Tomita-Takesaki theory. This theory will yield the mathematical objects that appear in the main result of this monograph. We follow the approach of \cite{Duvenhage1999} and \cite{Rieffel1977} to avoid encountering unbounded operators and domain issues. Finally, in chapter \ref{chp:final} we gather the partial result obtained in chapters \ref{chp:algebra}, \ref{chp:KMS}, and \ref{chp:tomita} and develop the final result which amounts to the connection between KMS states and Tomita-Takesaki theory.  

   