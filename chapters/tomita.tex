We are now going to develop the theory of Tomita-Takesaki following the approach of \cite{Duvenhage1999} and \cite{Rieffel1977}. This approach is appropriate for our purposes since it allows us to introduce the main objects of the theory, the Tomita-Takesaki theory, and the connection to KMS states, while avoiding unbounded operators (and therefore domain issues) by making use of real Hilbert spaces. The study of this theory is mathematically important due to its use in the classification of type III factors (and therefore physically important). As we will see, it is also physically important due to the interpretation we will give to $\Delta^{it}$ and the modular group in connection to KMS states.

For the remaining of the discussion I will assume knowledge of real Hilbert spaces, Borel functional calculus and polar decomposition. A quick account of this matters may be found in \cite{Duvenhage1999}. A more detailed exposition of Borel calculus and polar decompositions may be found in \cite{Rudin1991}.

\section{Operators of the Theory}

Let $\mathcal{K}$ and $\mathcal{L}$ be closed subspaces of a real Hilbert space $\mathcal{H}$ such that $\mathcal{K}\cap\mathcal{L}=\{0\}$ and $\mathcal{K}+\mathcal{L}$ is dense in $\mathcal{H}$. Define $P$ and $Q$ as the orthogonal projections on $\mathcal{K}$ and $\mathcal{L}$ respectively. Define $R=P+Q$ and let $JT=P-Q$ the polar decomposition of $P-Q$. We have that $P,Q,$ and $T$ are positive. We will start by proving some basic facts about this operators which we will use throughout the development of the theory. The proofs depend heavily on the Polar Decomposition theorem.

\begin{theorem}
\begin{enumerate}
\item $R$ and $2-R$ are injective and $2\geq R \geq 0$;
\item $T=R^{1/2}(2-R)^{1/2}$ and is injective;
\item $J$ is self-adjoint, isometric, $J^2=1$ and is bijective;
\item $T$ commutes with $P,Q,R,$ and $J$;
\item $JP=(1-Q)J$, $JQ=(1-P)J$, and $JR=(2-R)J$.
\item $J\mathcal{K}=\mathcal{L}^\bot$
\end{enumerate}
\end{theorem}

\begin{proof}
\begin{enumerate}
\item It is clear that the sum of positive operators is positive. Therefor $R\geq 0$. Moreover $1-P$ and $1-Q$ are also projections so that $2-R=(1-P)+(1-Q)\geq 0$. We conclude $2\geq R$. \\
Let $x\in\ker R$. Then
\begin{align}
\begin{split}
\|Px\|^2+\|Qx\|^2&=\langle Px,Px \rangle + \langle Qx,Qx\rangle = \langle x,P^*Px\rangle +\langle x, Q^*Qx\rangle \\
&=\langle x, P^2x\rangle + \langle x, Q^2x\rangle = \langle x,Px\rangle + \langle x,Qx\rangle \\
&=\langle x,Rx\rangle = 0
\end{split}
\end{align}
Therefore $x\in\ker P\cap\ker Q=\mathcal{K}^\bot\cap\mathcal{K}^\bot=(\mathcal{K}+\mathcal{L})^\bot=\mathcal{H}^\bot=\{0\}$ and we conclude $\ker R = {0}$, that is, $R$ is injective. Notice now that $\mathcal{K}^\bot$ and $\mathcal{L}^\bot$ satisfy the same properties that we imposed on $\mathcal{K}$ and $\mathcal{L}$ at the beginning. That is $\mathcal{K}^\bot\cap\mathcal{L}^\bot=\{0\}$ and
\begin{equation}
\overline{\mathcal{K}^\bot+\mathcal{L}^\bot}=(\mathcal{K}^\bot+\mathcal{L}^\bot)^{\bot\bot}=(\mathcal{K}^{\bot\bot}\cap\mathcal{L}^{\bot\bot})^\bot = (\mathcal{K}\cap\mathcal{L})^\bot=\{0\}^\bot = \mathcal{H}.
\end{equation}
Since $1-P$ and $1-Q$ are the projections on $\mathcal{K}^\bot$ and $\mathcal{L}^\bot$ we have by repeating the previous arguments that $2-R$ is injective.
\item We first note that 
\begin{align}
\begin{split}
T^2&=(P-Q)^*(P-Q)=(P-Q)(P-Q)=P-PQ-QP+Q \\
&=(P+Q)(2-P-Q)=R(2-R)=(R^{1/2})^2((2-R)^{1/2})^2 \\
&=(R^{1/2}(2-R)^{1/2})^2
\end{split}
\end{align}
Therefore, if $x\in \ker T$ the $x\in\ker T^2=\ker R(2-R)=\{0\}$. We conclude $T$ is injective. Since the product of positive elements is positive we have $T=R^{1/2}(2-R)^{1/2}$ by the uniqueness of the positive square root.
\item $J$ is an isometry on the orthogonal complement of
\begin{align}
\begin{split}
\ker J&=\ker (P-Q)\subseteq\ker(P-Q)^2=\ker(P-Q)^*(P-Q) \\
&=\ker T^2=\ker R(2-R)=\{0\}.
\end{split}
\end{align}
We conclude that $J$ is an isometry and is injective. $J^*=J$ and $J^2=1$ follows from $P-Q$ being self adjoint and the properties of the polar decomposition. Since $J^2=1$ we have that it is surjective and we conclude it is a bijection.
\item $T$ commutes with $J$ because $P-Q$ is self-adjoint. We also have $T^2P=(P-Q)^2P=P-PQP=PT^2$ and therefore the positive square root must also commute $TP=PT$. Analogously $TQ=QT$ and one concludes that $T$ and $R$ commute.
\item Since $TJP=JTP=(P-Q)P=P-QP=(1-Q)(P-Q)=(1-Q)TJ=T(1-Q)J$ and $T$ is injective we have $JP=(1-Q)J$. We also have $PJ=(J^*P^*)^*=(JP)^*=((1-Q)J)^*=J(1-Q)$ and we can solve for $JQ=(1-P)J$. We conclude $JR=J(P+Q)=(2-R)J.$
\item Since $J$ is surjective ($J^2=1$) and $1-Q$ is the projection on $\mathcal{L}^\bot$ 
\begin{equation}
J\mathcal{K}=JP\mathcal{H}=(1-Q)J\mathcal{H}=(1-Q)\mathcal{H}=\mathcal{L}^\bot.
\end{equation}
\end{enumerate}
\end{proof}

To obtain now the operators of the theory let $\mathcal{K}$ be a closed real subspace of a Hilbert space $\mathcal{H}$ such that $\mathcal{K}\cap i\mathcal{K}=\{0\}$ and $\mathcal{K}+i\mathcal{K}$ is dense in $\mathcal{H}$. Then the replacement of $\mathcal{L}$ by $i\mathcal{K}$ allows us to define the operators $P,Q,R,$ and $T$ as functions on $\mathcal{H}$. We will now give some properties of these as functions on the complex structure.  

\begin{theorem}
\begin{enumerate}
\item $R,2-R,$ and $T$ are positive in $\mathcal{H}$
\item $J$ is a conjugate linear isometry in $\mathcal{H}$.
\item $\langle x, Jy\rangle=\langle y, Jx\rangle$ for all $x,y\in\mathcal{H}$ 
\item $(JAJ)^*=JA^*J$ for all $A\in\mathcal{B}(\mathcal{H})$
\end{enumerate}
\end{theorem}

\begin{proof}
\begin{enumerate}
\item We only need to prove these functions are $\mathcal{B}(\mathcal{H})$. Note that for this proving linearity is enough since the topologies in the real and complex case coincide. Let $x\in\mathcal{H}$ and since $\mathcal{H}_\mathbb{R}=\mathcal{K}\oplus\mathcal{K}^\bot$ there exist unique $y\in\mathcal{K}$ and $z\in\mathcal{K}^\bot$ such that $x=y+z$. It is clear that $iz\in(i\mathcal{K})^\bot$ and therefore $iPx=iy=Q(ix)$. We conclude that $iP=Qi$ and therefore $Rix=P(ix)+Q(ix)=iQx+iPx=iRx$. We conclude that $R$ is linear. It is the clear too that $2-R$ is also linear. By uniqueness of the positive square root that $T=R^{1/2}(2-R)^{1/2}$ is also linear.
\item Given that $T$ is injective and 
\begin{align}
\begin{split}
TJ(ix)=(P-Q)(ix)=iQx-iPx=-i(P-Q)x=T(-iJx)
\end{split}
\end{align}
for all $x\in\mathcal{H}$, we conclude that $J$ is conjugate linear. Moreover, since the norm in $\mathcal{H}_\mathbb{R}$ is the same as $\mathcal{H}$ we have that $J$ is an isometry.
\item Let $x,y\in\mathcal{H}$. Then
\begin{align}
\begin{split}
\langle x,Jy\rangle &= \langle x,Jy\rangle_\mathbb{R}-i\langle x,iJy\rangle_\mathbb{R} \\
&= \langle x,Jy\rangle_\mathbb{R}-i\langle x,J(-iy)\rangle_\mathbb{R} \\
&= \langle Jx,y\rangle_\mathbb{R}-i\langle Jx,-iy\rangle_\mathbb{R} \\
&= \langle y,Jx\rangle_\mathbb{R}-i\langle -iy,Jx\rangle_\mathbb{R} \\
&= \langle y,Jx\rangle_\mathbb{R}-i\langle y,iJx\rangle_\mathbb{R} \\
&= \langle y,Jx \rangle.
\end{split}
\end{align}
\item Let $x,y\in\mathcal{H}$. Then
\begin{align}
\begin{split}
\langle x,JAJy \rangle &= \langle AJy, Jx\rangle = \langle Jy, A^*Jx\rangle \\
&= \overline{\langle A^*Jx,Jy\rangle}=\overline{\langle y, JA^*Jx\rangle} \\
&= \langle JA^*Jx,y\rangle
\end{split}
\end{align}
\end{enumerate}
We conclude that $(JAJ)^*=JA^*J$.
\end{proof}

\begin{theorem}
The spectra of $2-R$ and $R$ are equal.
\end{theorem}

\begin{proof}
We have that $J$ is a bijection and for all $\lambda\in\mathbb{C}$
\begin{equation}
J(\lambda-R)J=J\lambda J-JRJ=J^2\lambda-(2-R)J^2=\lambda-(2-R).
\end{equation}
Therefore, $\lambda-R$ is bijective if an only if $\lambda-(2-R)$ is. Moreover, by the open mapping theorem this implies that $\lambda-R$ is invertible by a bounded operator if and only if $\lambda-(2-R)$ is. We conclude $\sigma(R)=\sigma(2-R)$. 
\end{proof}

Since the spectrum of a positive operator $A\in\mathcal{B}(\mathcal{H})$ is in $[0,\infty)$ and compact and the function
\begin{align}
\begin{split}
[0,\infty)&\rightarrow \mathbb{C} \\
\lambda&\mapsto\lambda^z
\end{split}
\end{align}
(and $0^0=0$)is bounded and measurable we have that $A^z\in\mathcal{B}(\mathcal{H})$ is bounded. 

\begin{theorem}
We have that $JR^{it}J=(2-R)^{-it}.$
\end{theorem}

\begin{proof}
We have that if $P_R$ is the resolution of the identity for $R$,
\begin{align}
\begin{split}
2-R&=2\cdot1-R=2\int dP_R - \int \id_\mathbb{C}dP_R=\int 2 dP_R - \int \id_\mathbb{C}dP_R \\
&=\int (2-\id_\mathbb{C}) dP_R, 
\end{split}
\end{align}
from which we conclude $2-R=(2-\id_\mathbb{C})(R)$.
Therefore 
\begin{equation}
(2-\id_\mathbb{C})\sigma(R)=\sigma((2-\id_\mathbb{C})(R))=\sigma(2-R)=\sigma(R)
\end{equation}
and we have that $2-\id_\mathbb{C}$ is an homeomorphism from $\sigma(R)$ onto itself. We may therefore define the function
\begin{align}
\begin{split}
F:\Sigma&\rightarrow L(\mathcal{H}) \\
E&\mapsto JP_R((2-\id_\mathbb{C})(E))J
\end{split}
\end{align}
where $\Sigma$ is the Borel $\sigma$-algebra on $\sigma(R)$. We are now gonna prove this is a spectral valued measure.

Calculating we have $F(\varnothing)=JP_R(\varnothing)J=J0J=0$ and $F(\sigma(R))=JJ=1$. Let $E,D\in\Sigma$. Then
\begin{align}
\begin{split}
F(E\cap D)&=JP_R((2-id_\mathbb{C})(E\cap D))J \\
&=JP_R((2-id_\mathbb{C})(E)\cap(2-\id_\mathbb{C})(D))J \\
&=JP_R((2-id_\mathbb{C})(E))P_R((2-\id_\mathbb{C})(D))J \\
&=JP_R((2-id_\mathbb{C})(E))JJP_R((2-\id_\mathbb{C})(D))J \\
&=F(E)F(D)
\end{split}.
\end{align}
Finally, let $x,y\in\mathcal{H}$. Then for all $E\in\Sigma$
\begin{align}
\begin{split}
F_{x,y}(E)&=\langle x, F(E) y\rangle = \langle x, JP_R((2-\id_\mathbb{C})(E))Jy\rangle \\
&=\langle P_R((2-\id_\mathbb{C})(E))Jy,Jx\rangle \\
&=\langle Jy,P_R((2-\id_\mathbb{C})(E))Jx\rangle =P_{R_{Jy,Jx}}((2-\id_\mathbb{C})(E))
\end{split}
\end{align}
and we conclude $F_{x,y}=P_{R_{Jy,Jx}}\circ (2-\id_\mathbb{C})$ and since $(2-\id_\mathbb{C})$ is a bijection on $\sigma(R)$, $F_{x,y}$ is a complex measure.
Now, notice that for all $x,y\in\mathcal{H}$ we have that
\begin{align}
\begin{split}
\langle x, Ry \rangle &= \langle x, J(2-R)J y\rangle = \langle (2-R)J y, Jx \rangle \\
&= \langle Jy, (2-R)Jx \rangle = \left\langle Jy, \int (2-\id_\mathbb{C})dP_R Jx\right\rangle \\
&= \int (2-\id_\mathbb{C})dP_{R_{x,y}}
\end{split}
\end{align}
and since $2-\id_\mathbb{C}$ is a bijection,
\begin{align}
\begin{split}
\langle x, Ry\rangle &= \int (2-\id_\mathbb{C})\circ (2-\id_\mathbb{C})d(P_{R_{x,y}}\circ (2-\id_\mathbb{C})) \\
&= \int \id_\mathbb{C} d(P_{R_{x,y}}\circ (2-\id_\mathbb{C})) = \int \id_\mathbb{C} dF_{x,y}.
\end{split} \\
&= \left\langle x, \int \id_\mathbb{C} dF y\right\rangle
\end{align}
Therefore, by the uniqueness of the spectral resolution guaranteed by the spectral theorem we have that $P_R=F$. We may repeat a similar analysis to show that
\begin{align}
\begin{split}
G:\Sigma&\rightarrow L(\mathcal{H}) \\
E&\mapsto P_R((2-\id_\mathbb{C})(E))
\end{split}
\end{align}
is a resolution of the identity on $\sigma(R)$ and it is clear that $G_{x,y}=P_{R_{x,y}}\circ (2-\id_\mathbb{C})$. Notice that for all $x,y\in\mathcal{H}$
\begin{align}
\begin{split}
\langle x, (2-R) y \rangle &= \left\langle x, \int (2-\id_\mathbb{C})dP_{R}y\right\rangle = \int (2-\id_\mathbb{C})dP_{R_{x,y}} \\
&= \int (2-\id_\mathbb{C})\circ (2-\id_\mathbb{C}) d(P_{R_{x,y}}\circ (2-\id_\mathbb{C})) \\
&= \int \id_\mathbb{C} dG_{x,y},
\end{split}
\end{align}
that is, $G$ is the resolution of the identity of $2-R$.
Finally, we calculate for all $x,y\in\mathcal{H}$ and $t\in\mathbb{R}$
\begin{align}
\begin{split}
\langle x, JR^{it}J y\rangle &= \langle R^{it}Jy,Jx\rangle = \langle Jy,R^{-it}Jx\rangle \\
&= \left\langle Jy, \int \lambda^{-it}dP_R(\lambda)Jx\right\rangle = \int \lambda^{-it}dP_{R_{Jy,Jx}} \\
&= \int (2-\lambda)^{-it}d(P_{R_{Jy,Jx}}\circ (2-\id_\mathbb{C}))(\lambda) \\
&= \int (2-\lambda)^{-it}dF_{x,y}(\lambda) = \int (2-\lambda)^{-it}dE_{x,y}(\lambda) \\
&= \int (2-\lambda)^{-it}d(E_{x,y}\circ\id_\mathbb{C})(\lambda) \\
&= \int (2-\lambda)^{-it}d(E_{x,y}\circ (2-\id_\mathbb{C})\circ (2-\id_\mathbb{C}))(\lambda) \\
&= \int (2-\lambda)^{-it}d(G_{x,y}\circ (2-\id_\mathbb{C}))(\lambda) \\
&= \int \lambda^{-it}d(G_{x,y})(\lambda) = \langle x, (2-R)^{-it}y\rangle.
\end{split}
\end{align}
We conclude $JR^{it}J=(2-R)^{-it}$.
\end{proof}

We are now ready to define the main operator in the theory.

\begin{definition}
For all $t\in\mathbb{R}$ define $\Delta_t = (2-R)^{it}R^{-it}$.
\end{definition}